% engine=luatex

\usemodule[t][presentation]
\usemodule[mathsets]

\def\LadderBreak{36}

\environment macros

\setupTitle
  [title={Optimal design of sequential teams},
   author={Aditya Mahajan \crlf Dept of Electrical Engineering \crlf
           Yale University},
   date={Joint work with:  \crlf
    Ashutosh Nayyar and Demos Teneketzis (UofM); Sekhar Tatikonda (Yale) \blank[medium] Presented at: McGill University, May 20, 2009 \phantom{A} \crlf \phantom{Presented at:}Queen's University, May 22, 2009 }]

\starttext

\placeTitle

\SlideTitle Outline of the talk

\startitemize[n]
  \head Overview of multi-agent systems 

    \vskip -1.5\baselineskip
    \startitemize[2]
      \item Classification :
             games vs.\ teams 
             and single-stage vs.\ multi-stage
      \item Sequential decomposition : what and why
    \stopitemize

  \head Sequential teams with private and common observations

    \vskip -1.5\baselineskip
    \startitemize
      \item A coordinator of a decentralized system
      %\item The notion of information state
      \item Sequential decomposition for finite and infinite horizon 
    \stopitemize

  \head An example: multi-access broadcast

  \head A graphical model for sequential teams 

    \vskip -1.5\baselineskip
    \startitemize
      \item Automated method to find structural results
    \stopitemize

\stopitemize

\SlideTitle Multi-agent decentralized systems

\startitemize

  \head Applications

    \startitemize[columns,two]
      \item communication networks
      \item sensor networks
      \item surveillance networks
      \item transportation networks
      \item control systems
        \column
      \item monitoring and diagnostic systems
      \item multi-robot systems
      \item multi-core CPUs
      \item \unknown
    \stopitemize

  \head Salient features

    \startitemize
      \item System has different components
      \item These components know different information
      \item The components need to cooperate and coordinate
    \stopitemize

\stopitemize

\Interlude \important{Question} \crlf
    How do we approach the design of a decentralized multi-agent system?

\SlideTitle Classification of multi-agent systems

\startitemize[n]

  \item Teams  vs.\ Games

  \item Single-stage  vs.\ Multi-stage

    \startitemize[unpacked,1]
      \item Sequential  vs.\ non-sequential systems

        \startitemize[unpacked,2]
          \item Classical  vs.\ non-classical  information structures
        \stopitemize
    \stopitemize

    \startimportant
      Sequential multi-stage teams with non-classical information structures
    \stopimportant

\stopitemize

\SlideTitle Sequential Decomposition

\startitemize

  \head Divide and conquer :  \crlf \null \qquad 
      Exploit sequential and multi-stage nature of the problem 
    
      \startimportant
        Convert a one-shot optimal design problem into a sequence of nested
        optimization problems
      \stopimportant


  \head Classical information structure (MDP, POMDP) 

    \vskip -1.5\baselineskip
    \startitemize
      \item Dynamic programming
    \stopitemize

\stopitemize

\vfill

\SlideTitle Why consider sequential decomposition

\startitemize
  \head Finite horizon

    \startitemize
      \item Brute force search always possible but has high complexity
      \head Provides a \emph{systematic} way to search for an optimal solution
        \emph{efficiently}

    \stopitemize

  \head Infinite horizon

    \startitemize
      \item Brute force search not possible
      \item An arbitrary solution cannot be implemented
      \head Identify qualitative properties to search and implement optimal
        designs \emph{compactly}

    \stopitemize

  \head May help in identifying (and proving) other qualitative properties

\stopitemize

\SlideTitle Sequential Decomposition

\startitemize
  \head Non-classical information structure

    \vskip -1.5\baselineskip
    \startitemize[paper,unpacked]
      \item Hans S.\ Witsenhausen, 
          \emph{A standard form for sequential
          stochastic control}, \crlf
          Math.\ Systems Theory, 7 (1973), pp.~5--11.
          \crlf     
          \null \hfill \important
          {Does not work for infinite horizon problems}

      \item Aditya Mahajan, 
         \emph{Sequential decomposition of
         sequential teams}, \crlf
         Ph.D.\ Thesis, University of Michigan, Ann Arbor
         (2008).
          \crlf     
         \null \hfill \important
         {Not easy to extend to more than two agents}
    \stopitemize
\stopitemize


\starthiding %{{{
\SlideTitle Finite horizon problems

\adaptlayout[lines=+2]

\startitemize
  \head MDP (Markov decision process)

    \startitemize
      \head Qualitative property : Take actions based only on the current
        state

        \starttabulate[|l|l|]
          \NC Brute force \EQ  $\SIZE U^{(\SIZE X × \SIZE U)^T}$ \NC \NR
          \NC Sequential decomposition \EQ $T × \SIZE X × \SIZE U^2$ \NC \NR
        \stoptabulate
    \stopitemize

  \head POMDP (Partially observable Markov decision process)

    \startitemize
      \head Qualitative property : Take actions based only on the controller's
        belief about the state of the system

        \starttabulate[|l|p|]
          \NC Brute force \EQ $\SIZE U^{(\SIZE Y × \SIZE U)^T}$ \NC \NR
          \NC Sequential decomposition \EQ Worst case (for exact solution)
              $\SIZE U^{\SIZE Y^T}$ \NC \NR
          \NC \NC May be polynomial for specific instances \NC \NR
          \NC \NC Under some conditions approximation algorithms do better \NC
              \NR
        \stoptabulate
    \stopitemize
\stopitemize


\SlideTitle Infinite horizon problems

\startitemize
  \head MDP (Markov decision process)

    \startitemize
      \item Qualitative property : Take actions based only on the current
        state

      \head Can restrict attention to time invariant designs

      \item Optimal design can be found by solving a linear program with $\SIZE
        X$ variables and $\SIZE X × \SIZE U$ constraints

    \stopitemize

  \head POMDP (Partially observable Markov decision process)

    \startitemize
      \item Qualitative property : Take actions based only on the controller's
        belief about the state of the system

      \head Can restrict attention to time invariant designs

      \item Optimality equations can be approximately solved using randomized
        algorithms whose worst case complexity is polynomial in $\SIZE X$ and
        $\SIZE U$

    \stopitemize

\stopitemize
\stophiding %}}}


\Interlude  Sequential decomposition \\ of
\important{multi-agent} sequential teams 
that works for \\
\important{finite and
infinite horizon}

% \Interlude {\switchtobodyfont[120pt]\important{YES}}

\SlideTitle Models considered in this talk 

\adaptlayout[lines=+2]

\startitemize[n]

  \head multi-agent ($> 2$) teams with private and common observations

    \startitemize

      \head Assume structural property of optimal controller

      \item Identify a \emph{coordinator} of the system

      \item Identify an \emph{information state} sufficient for performance
        analysis

      \item Optimally control the evolution of information state 

        \startlinecorrection
          \midaligned{\bfc$\Updownarrow$\hskip 4em}
        \stoplinecorrection

        Markov decision process where
        \vskip -.5\baselineskip
          \starttabulate[|i7l|p|] 
            \NC State space  \EQ space of probability measures \NC \NR
            \NC Action space \EQ space of functions \NC \NR
          \stoptabulate

    \stopitemize

  \head A graphical model for sequential teams

    \startitemize
      \item Automated method to find structural results
    \stopitemize

\stopitemize

\SlideTitle Main results

\startitemize
  \head Multi-agent teams : finite horizon

    \startitemize

      \item Coordinator's problem is a centralized problem
      \item Absence of any common observation $\implies$ 
        coordinator = designer

    \stopitemize

  \head Multi-agent teams : infinite horizon

    \startitemize

      \item Cannot restrict attention to time invariant designs

      \item Can restrict attention to time invariant \important{meta-designs}

    \stopitemize

  \head Graphical model for sequential teams

    \startitemize

      \item Structural results can be derived based on conditional independence

      \item Conditional independence can be tested efficiently in graphical
        models

    \stopitemize

\stopitemize

\Interlude The first model \unknown{}

\SlideTitle Teams with common information

\startitemize
  \head Some models of teams considered in the literature \unknown {}

    \startitemize[columns, two]
      \item delayed state sharing
      \item delayed information sharing
      \item delayed observation sharing
      \item delayed control sharing
      \item delayed belief sharing 
        \column
      \item periodic state sharing
      \item periodic information sharing
      \item periodic observation sharing
      \item periodic control sharing
      \item periodic belief sharing 
    \stopitemize
\startimportant
  Sharing is good! Can we generalize this?
\stopimportant
\stopitemize

\SlideTitle Teams with common information

\startitemize
  \item \emph{Plant:} \quad $\VAR x(t+1) = f(\VAR x(t), \VAR u[1:n](t), \VAR w(t))$
  \item \emph{Observations}

    \vskip -.5\baselineskip
    \startitemize
      \item {Common Observation:} \quad 
        $\VAR z(t) = o(\VAR x(t), \VAR u[1:n](t-1), \VAR q(t))$
      \item {Private Observation:} \quad
        $\VAR y[i](t) = h_i(\VAR x(t), \VAR n[i](t))$
    \stopitemize
  \item \emph{Control at agent $i$:} \quad
      $\VAR u[i](t) = g_{i,t}(\VAR y[i](1:t), \VAR u[i](1:t-1), \VAR z(1:t))$

  \item \emph{Design:} \quad $G_T \DEFINED \VAR g[1:n](1:T)$ 
    \quad (control laws of all agents for all time)

  \item \emph{Cost} 
    
    \vskip -.5\baselineskip
    \startitemize
      \item {At time $t$:} \quad $c(\VAR x(t), \VAR u[1:n](t))$
      \item {Of a design $G_T$:} \quad
        $\displaystyle \COST_T(G_T) =
        \EXP^{G_T}{\sum_{t=1}^T c(\VAR x(t), \VAR u[1:n](t))}$
    \stopitemize

\stopitemize

\SlideTitle Problem P (Finite horizon)

\startitemize 
  \head Given

    \startitemize[columns,two]
      \item Alphabets $\ALPHABETS[X,Z,Y_i,U_i]$
      \item Plant function $f$
      \item Observation functions $o$ and $h_i$
      \item Cost functions $c$
      \item PMF of $\VAR w(t)$, $\VAR q(t)$, and $\VAR n[i](t)$
    \stopitemize

  \head Determine

    \startitemize
      \item An optimal design $G^*_T$
        \startformula
          \COST_T(G^*_T) = J^*_T \DEFINED
          \min_{G_T} \COST_T(G_T)
        \stopformula
    \stopitemize
\stopitemize

\SlideTitle Salient Features

\startitemize
  \head Team problem

    All agents have common objective 

  \head Sequential team

    Agent's actions or events in nature do not influence order of agent's
    actions.

  \head Data, information and non-classical information structure 

    \centerline{$σ(\VAR y[i](1:t), \VAR u[i](1:t-1), \VAR z(1:t)) 
    \mathrel{\cancel{\subset \atop \supset}}
    σ(\VAR y[j](1:t), \VAR u[j](1:t-1), \VAR z(1:t))$, \quad $i \neq j$}

\stopitemize

\Interlude Solution Approach: \\ Sequential Decomposition 

\SlideTitle Assumption on controller structure

\startitemize
  \head Assumption A

   \vskip -1\baselineskip
    \startimportant[before=\blank, after=\blank, width=broad]
      Without loss of optimality, 
      all controllers can \emph{compress} their past private information 
      $(\VAR y[i](1:t-1), \VAR u[i](1:t-1))$ into a sufficient statistic (memory)
      $\VAR m[i](t)$ that takes values in a \emph{time invariant} space 
      $\ALPHABET M_i$.
    \stopimportant
   \vskip -1\baselineskip

  \head Modified control

    \centerline{$\displaystyle 
    \VAR u[i](t) = 
    \cross{g_{i,t}(\VAR y[i](1:t), \VAR u[i](1:t-1), \VAR z(1:t))} = 
    g_{i,t}(\VAR y[i](t), \VAR m[i](t), \VAR z(1:t))$}

  \head Memory update

    \centerline{$\displaystyle 
    \VAR m[i](t+1) = l_{i,t}(\VAR y[i](t), \VAR m[i](t), \VAR z(1:t))$}

  \head Many systems satisfy Assumption~A

\stopitemize

\SlideTitle Solution Approach --- An alternate problem

\startitemize
  \head Problem PC (Problem with a coordinator)

    A \emph{centralized} system with $n$ components and a coordinator.

  \item \emph{Coordinator's Observations}:  $\VAR z(t)$

  \item \emph{Coordinator's {\em meta-control} }  \crlf
    \null \qquad
    $γ_{i,t} = \ALPHABET Z^t \to (\ALPHABET Y_i × \ALPHABET M_i \to
    \ALPHABET U_i)$ 
    and 
    $λ_{i,t} = \ALPHABET Z^t \to (\ALPHABET Y_i × \ALPHABET M_i \to
    \ALPHABET M_i)$ 
    \startformula \startalign[m=2,distance=4em]
      \NC \hat g_{i,t} \EQ γ_{i,t}(\VAR z(1:t)) 
      \NC \hat l_{i,t} \EQ λ_{i,t}(\VAR z(1:t)) \NR
    \stopalign \stopformula

  \item \emph{Component's operation:} \crlf
    \null \qquad
    Passively applies the functions supplied
    by the coordinator.
    \startformula \startalign[m=2,distance=4em]
      \NC \VAR u[i](t)  \EQ \hat g_{i,t}(\VAR y[i](t), \VAR m[i](t))
      \NC \VAR m[i](t+1)\EQ \hat l_{i,t}(\VAR y[i](t), \VAR m[i](t)) \NR
    \stopalign \stopformula
    
\stopitemize

\SlideTitle The two problems are equivalent

\startitemize
  \head $\COST_T(\text{Problem PC}) ≥ \COST_T(\text{Problem P})$

    \startitemize
      \item Given any strategy $(G_T, L_T)$ of controllers, an equivalent coordinator
        meta-strategy $(Γ_T, Λ_T)$ can be constructed.

        \startformula \startalign[m=2,distance=4em]
          \NC γ_{i,t}(\VAR z(1:t)) \EQ g_{i,t}(⋅, ⋅, \VAR z(1:t)) 
          \NC λ_{i,t}(\VAR z(1:t)) \EQ l_{i,t}(⋅, ⋅, \VAR z(1:t)) \NR
        \stopalign \stopformula
    \stopitemize

  \head $\COST_T(\text{Problem P}) ≥ \COST_T(\text{Problem PC})$

    \startitemize
      \item Coordinators actions are measurable at all agents.
      \item All agents can implement the coordinator's meta-strategy.
        \startformula \startalign
          \NC g_{i,t}(\VAR y[i](t),\VAR m[i](t), \VAR z(1:t)) 
              \EQ γ_{i,t}(\VAR z(1:t))(\VAR y[i](t), \VAR m[i](t)) \NR
          \NC l_{i,t}(\VAR y[i](t),\VAR m[i](t), \VAR z(1:t)) 
              \EQ λ_{i,t}(\VAR z(1:t))(\VAR y[i](t), \VAR m[i](t)) \NR
        \stopalign \stopformula
    \stopitemize

\stopitemize

\Interlude The coordinator's problem is centralized!

\SlideTitle Information states:
  Pr( state $\mathsurround\zeropoint\bigg|$ information )

\startitemize 
  \let\1\important
  
  \head State (for coordinator)

    \startformula \startalign[n=5,align={right,left,middle}]
      \NC \VAR s[1](t) \EQ (\VAR x(t), \NC \VAR y[1:n](t), \NC \VAR u[1:n](t-1), 
      \NC \VAR m[1:n](t-1)) \NR
      \NC \VAR s[i](t) \EQ (\VAR x(t), \NC \VAR y[1:\1{k}](t), 
      \NC \VAR u[1:\1{i-1}](t), 
      \NC \VAR m[1:\1{i-1}](t), \VAR m[\1{i}:n](t-1)), \quad i = 2, \dots, n \NR
    \stopalign \stopformula

   \emph{Cost}: \quad
      $c(\VAR x(t), \VAR u[1:n](t)) = \hat c(\VAR s[n](t))$

  \head Information (at the coordinator)
    
    \startformula
      σ(\VAR z(1:t);\underparent{
      \VAR {\hat g}[1:n](1:t-1), \VAR {\hat l}[1:n](1:t-1),
      \VAR {\hat g}[1:i-1](t), \VAR {\hat l}[1:i-1](t)}_{\VAR φ[i](t)},
      \cross{\VAR γ[1:n](1:t-1)}, \cross{\VAR λ[1:n](1:t-1)})
    \stopformula

  \head Information state

    \startformula
      \VAR π[i](t)(\VAR z(1:t), \VAR φ[i](t)) 
      = \PR{\VAR S[i](t) | σ(\VAR z(1:t); φ(t))} = 
      \PR^{\VAR φ[i](t)}{\VAR S[i](t) | \VAR Z(1:t) = \VAR z(1:t)}
    \stopformula
\stopitemize 


\SlideTitle Properties of information states

\startitemize
  \head Update

    There exists functions $F_i$, $i=1,\dots,n$, such that
    \startformula \startalign[m=2]
      \NC \VAR π[1](t) \EQ F_1(\VAR π[n](t-1), \VAR {\hat g}[k](t-1),
          \VAR {\hat l}[k](t-1)) \NC \NC \NR
      \NC \VAR π[i](t) \EQ F_1(\VAR π[i-1](t), \VAR {\hat g}[i-1](t),
          \VAR {\hat l}[i-1](t)), \NC  i \EQ 2, \dots, n \NR
    \stopalign \stopformula

  \item Information state update is independent of the \important{meta-strategy}

  \item Information state is measurable at all controllers

  \item Information state is \quotation{time-invariant}
\stopitemize

\SlideTitle Sequential Decomposition

\startitemize
  \head Initialization

    \startformula
      V_{n,T}(\tilde π_n) = 
      \EXP{\hat c(\VAR S[n]() | \VAR π[n](T) = \tilde π_n)}
    \stopformula

  \head Backward recursion

    \startitemize 
      \item For $i = 1,\dots, n-1$, and $t = 1,\dots, T$,
        \startformula
          V_{i,t}(\tilde π_i) = \inf_{\hat g_{i,t}, \hat l_{i,t}} \big[
            \EXP^{\hat g_{i,t}, \hat l_{i,t}}
            {V_{i+1, t}(\VAR π[i+1](t)) | \VAR π[i](t) = \tilde π_i}
            \big]
        \stopformula

      \item For $t=1,\dots,T$,
        \startformula
          V_{n,t}(\tilde π_n) = 
      \EXP{\hat c(\VAR S[n]() | \VAR π[n](t) = \tilde π_n)} + 
          \inf_{\hat g_{n,t}, \hat l_{n,t}} \big[
            \EXP^{\hat g_{n,t}, \hat l_{n,t}}
            {V_{1, t+1}(\VAR π[1](t+1)) | \VAR π[n](t) = \tilde π_n}
            \big]
        \stopformula

    \stopitemize

\stopitemize

\SlideTitle An interpretation of the solution

\startitemize
  \item All the agents decide what to do before the system starts operating

    \startimportant
      In the presence of common information, the coordinator allows the agents
      to  adapt to the
      component of the same path that is commonly observed
    \stopimportant

    \startimportant
      In the absence of common information, the coordinator is same as the
      system designer
    \stopimportant

  \item Information states store and  process the common information and past
    policies more efficiently.

\stopitemize

\SlideTitle Infinite horizon problem

\startitemize
  \item  Information states belong to \quotation{time invariant} spaces, so the
    sequential decomposition extends to infinite horizon.

  \item Meta-strategy is stationary $\implies$ meta control laws are time
    invariant.

  \item Actual control laws change with time

  \item The coordinator's viewpoint makes it easy to find and implement optimal
    time varying control laws.

\stopitemize

\SlideTitle Summary so far \unknown {}

\startitemize
  \item Started with a general $n$ agent sequential team with common
    observations.
    
  \item Assume certain structural results exist (private information can be
    compressed to a sufficient statistic)

  \item Consider an alternate problem (with passive agents and a coordinator)

  \item Both problems are equivalent

  \item The coordinator's problem is centralized!

  \item Coordinator chooses meta control laws (control laws that choose control
    laws)

  \item Each step of the sequential decomposition is a functional problem
\stopitemize

\Interlude An example 

\SlideTitle Multiaccess broadcast

    \placefigure[right,none]{}
      {\externalfigure[network][width=0.4\textwidth, location=middle]}

\SlideTitle Multiaccess broadcast

    \placefigure[right,none]{}
      {\externalfigure[network][width=0.4\textwidth, location=middle]}
\startitemize
  \head MAB Channel

    \vskip -1\baselineskip
    \startitemize
      \item Single user transmits $\implies$ 
        \externalfigure[face-smile][location=middle]
      \item Both users transmit   $\implies$ 
        \externalfigure[face-sad][location=middle]
    \stopitemize


  \head Transmitters

    \startitemize
      \item Queues with buffer of size 1
      \item Packet held in queue until successful transmission
      \item Packet arrival is independent Bernoulli process
    \stopitemize
\stopitemize

\SlideTitle Multiaccess broadcast

\startitemize
  \head Channel Feedback

    \centerline{Both users know if there was no Tx, successful Tx, or collision}

  \head Policy of user

    \startformula
      \VAR u[i](t) = g_{i,t}(\VAR x[i](1:t), \VAR u[i](1:t-1), \VAR z(1:t))
    \stopformula

  \head Objective : Maximize throughput (or minimize delay)

    \startitemize
      \item Avoid collisions
      \item Avoid idle
    \stopitemize

\stopitemize

\SlideTitle History of multiaccess broadcast

\startitemize[paper]
  \item Hluchyj and Gallager, 
    \emph{Multiaccess of a
    slotted channel by finitely many users}, NTC~81.

    \startitemize[2]
        \vskip -\lineheight
      \item Considered symmetric arrival rates 
      \item Restricted attention to \quotation{window protocols}
    \stopitemize

  \item Ooi and Wornell, 
    \emph{Decentralized control of multiple access broadcast
    channels}, CDC~96.

    \startitemize[2]
        \vskip -\lineheight
      \item Considered a relaxation of the problem
      \item Numerically find optimal performance of the relaxed problem 
      \item Hluchyj and Gallager's scheme meets this upper bound
    \stopitemize

  \item AI Literature
    \startitemize
        \vskip -\lineheight
      \item Consider the case of asymmetric arrival rates
      \item Approximate heuristic solutions for small horizons
    \stopitemize

\stopitemize

\SlideTitle Optimal Solution

\startitemize
  \head Structural Result

    \startformula
      u_i(t) = g_{i,t}(\VAR x[i](t), \cross{\VAR x[i](1:t-1)}, 
      \cross{\VAR u[i](1:t-1)}, \VAR z(1:t))
    \stopformula

  \head Assumption A is satisfied $\implies$ sequential decomposition

  \head Further simplifications

    \startitemize
      \item Information states
        \startformula
          P( (\VAR x[1](t), \VAR x[2](t)) \mid \VAR z(1:t)) \equiv
          ( P(\VAR x[1](t) \mid \VAR z(1:t)), P(\VAR x[2](t) \mid \VAR z(1:t))
        \stopformula

      \item Actions: $\hat g_{i,t}(0) = 0$, $\hat g_{i,t}(1) = ??$
    \stopitemize

\stopitemize

\SlideTitle Optimal Solution


\startitemize
  \item Sequential decomposition same as that of a POMDP with finite state and
    action space.

  \head For symmetric arrival rates $p$

    \startitemize
      \item If $p > τ$, follow TDMA
      \item If $p < τ$, 
        \startitemize[n][left=S]
          \item If you have a packet, transmit it. If collision, one user moves
            to S2.
          \item Idle, then move to S1
        \stopitemize
    \stopitemize

  \item Same as the strategy proposed by Hluchyj and Gallager. \crlf
    \centerline{\emph{We can prove optimality.}}

  \item All previous attempts provide approximate solutions!

\stopitemize


\Interlude Back to general teams \\  Structural properties?

\SlideTitle A model for sequential team

\startitemize
  \head Components of a sequential team

    \startitemize[unpacked]
      \item A set $N$ of indices of system variables $\{X_n, n \in N\}$ and a partial order \rlap{$\prec$ on $N$}
        \vskip -.5\baselineskip
        \startitemize[packed]
          \item $A \subset N$, variables generated by DM \quad \symbol[3] \enspace
            $N \setminus A$, variables generated \rlap{by nature}
          \item $R \subset N$, reward variables
        \stopitemize

      \item Finite sets $\{\ALPHABET X_n$, $n \in N\}$ of state spaces of $X_n$

      \item $\{I_n$, $n \in N\}$, such that for all $i \in I_n$, $i \prec n$.
        $\ALPHABET I_n = \prod_{i \in I_n} \ALPHABET X_i$

      \item $F_{N\setminus A} = \{f_n$, $n \in N \setminus A\}$, where $f_n$ is a
        stochastic kernel from $\ALPHABET I_n$ to $\ALPHABET X_n$.

    \stopitemize

  \head Design

    \startitemize
      \item $G_A = \{g_n$, $n \in A\}$, where $g_n$ is a decision rule from
        $\ALPHABET I_n$ to $\ALPHABET X_n$
    \stopitemize

\stopitemize

\SlideTitle A model for sequential team

\startitemize
  \head Probability measure induced by a design

    \startformula
      P^{G_A}(X_N) = \prod_{n \in N\setminus A} f_n(X_n | I_n) \prod_{n \in A}
      \IND {X_n = g_n (I_n)}
    \stopformula

  \head Optimization problem

    Minimize $\displaystyle \EXP{\sum_{n \in R} X_n}$, where the expectation is
    with respect to $P^{G_A}$.
\stopitemize

\SlideTitle Generality of the Model

\startitemize
  \head Witsenhausen's intrinsic model 

    \startitemize[paper,unpacked]
      \item Hans S.\ Witsenhausen, 
        \emph{On information structures, feedback and causality}, \crlf
        SIAM Journal of Control, 9 (1971), pp.~149-160. \crlf
        \null \hfill \important
          {partial order $\iff$ sequentiality}

    %  \item Hans S.\ Witsenhausen, 
    %    \emph{The intrinsic model for stochastic control: Some open problems}
    %    \crlf
    %    Lecture Notes in Economics and Mathematical Systems, 17 (1975),
    %    pp~322-335.
    \stopitemize

  \head Witsenhausen's sequential control model 

    \startitemize[paper]
      \item Hans S.\ Witsenhausen, 
        \emph{A standard form for sequential stochastic control}, \crlf
        Math.\ Systems Theory, 7 (1973), pp.~5--11. 
    \stopitemize

  \head Witsenhausen's equivalent control model 

    \startitemize[paper]
      \item Hans S.\ Witsenhausen, 
        \emph{Equivalent stochastic control problems}, \crlf
        Math.\ Controls, Signals and Systems, 1 (1988), pp.~3--11.
    \stopitemize
\stopitemize
          
% \startitemize %{{{
%   \head Conditions to check if a multi-stage game or team is
% 
%     \startitemize
%       \item causal
%       \item deadlock free
%       \item sequential
%     \stopitemize
% 
%   \head Properties of such systems
% 
% \stopitemize
% 
% \SlideTitle How to check if a system is sequential \unknown {}
% 
% \startitemize
%   \head The intuitive definition
% 
%     DMs actions or events in nature do not influence order of DM's
%     actions.
% 
%   \head The (informal) formal definition
% 
%     \vskip -1\baselineskip
%     \startitemize
%       \item \emph{Subsystem}: A set of system variables form a subsystem if
%         these variables depend only on other variables in the set.
% 
%       \item \emph{Closure of a DM~$α$:} $\overline {\{α\}}$ is the smallest
%         subsystem containing $α$ (or more precisely, the data observed by~$α$).
% 
%       \item \emph{A binary relation on DMs:} $α \leftarrow β$ iff 
%         $\overline {\{α\}} \subseteq \overline {\{β\}}$.
% 
%       \item $\leftarrow$ is reflexive and transitive $\implies$ a
%         \important{quasi-order}.
% 
%     \stopitemize
% 
%     \startimportant
%       A system is sequential if and only if $\leftarrow$ is a partial order.
%     \stopimportant
% 
% \stopitemize %}}}

\SlideTitle A graphical model for sequential teams

\startitemize
  \head Directed Acyclic Factor Graph

  \head Factor Graph

    \startitemize
      \item Variable node $n$ $\equiv$ system variable $X_n$
      \item Factor   node $\tilde n$ $\equiv$ stochastic kernel $f_n$ or
        decision rule $g_n$
    \stopitemize

  \head Directed Graph

    \startitemize
      \item $(i, \tilde n)$, for each $n \in N$ and $i \in I_n$
      \item $(\tilde n, n)$, for each $n \in N$
    \stopitemize
      
  \head Acyclic Graph

    \startitemize
      \item Partial order on $N$ implies acyclic graph
    \stopitemize
\stopitemize

\SlideTitle An example: MDP

\startitemize 
  \head Mathematical Model

    \startitemize
      \item Plant:   $f_{t-1}(x_t | x_{t-1}, u_{t-1})$
      \item Control: $u_t = g_t(x_1, \dots, x_t, u_1, \dots, u_{t-1})$
      \item Reward:  $r_t = ρ_t(x_t, u_t)$
    \stopitemize

\stopitemize

\SlideTitle An example: MDP

\startitemize 
  \head Mathematical Model

    \startitemize
      \item Plant:   $f_{t-1}(x_t | x_{t-1}, u_{t-1})$
      \item Control: $u_t = g_t(x_1, \dots, x_t, u_1, \dots, u_{t-1})$
      \item Reward:  $r_t = ρ_t(x_t, u_t)$
    \stopitemize

  \head Graphical Model

    \vskip -2\baselineskip
    \placefigure[here,none]{}
      {\externalfigure[fig-mdp][page=1, width=0.8\textwidth]}

\stopitemize

\dostepwiserecurse{1}{3}{2}{
\SlideTitle Structural Results for MDP

\startimportant
  Without loss of optimality, $u_t = g_t(x_t)$
\stopimportant

\startitemize
  \head Graphically \par

    \vskip -2\baselineskip
    \placefigure[here,none]{}
      {\externalfigure[fig-mdp][page=#1, width=0.8\textwidth]}

\stopitemize
}

\SlideTitle Structural results

\startitemize
  \head The main idea

    \startalignment[middle]
      If some data available at a DM is independent of future rewards given the control
      action and other data at the DM, then that data can be ignored
    \stopalignment

    \startimportant
      Can we automate this process?
    \stopimportant

\stopitemize

\Interlude \switchtobodyfont[20pt] \setupinterlinespace
        Struct. result $\equiv$ cond.\ independence \blank[big]
        Graphical models can easily test conditional independence

\SlideTitle Graphical models

\startitemize
  \head Terminology

    \vskip -1\baselineskip
    \startitemize
      \item \emph{$\PARENTS{n}$}: All nodes $m$ such that $m \CONNECTED
        n$

      \item \emph{$\CHILDREN{n}$}: All nodes $m$ such that $n
        \CONNECTED m$

      \item \emph{$\ANCESTORS{n}$}: All nodes $m$ such that there is a
        directed path from \rlap{$m$ to $n$}

      \item \emph{$\DESCENDANTS{n}$}: All nodes $m$ such that there
        is a directed path from \rlap{$n$ to $m$}

    \stopitemize

  \head In terms of teams

    \vskip -1\baselineskip
    \startitemize
      \item Parents of a control (factor) node = data observed by controller
      \item Children of a control node = control action
      \item Ancestors of a control node = all nodes that affect the
        data observed 
      \item Descendants of a control node = all nodes that are affected
        by the control action
    \stopitemize

\stopitemize

\IncludePicture
  [horizontal]
  [fig-mdp]
  [page=3]
  {Graphical Models}

\SlideTitle Conditional independence 

\startitemize
  \head Three canonical graphs to verify $x \INDEPENDENT z \mid y$

    \vskip -3\baselineskip

    \placefigure[here,none]{}
    \startcombination[3]
      {\externalfigure[fig-ind][page=1]}
      {\externalfigure[face-smile][location=middle] Markov chain}
      {\externalfigure[fig-ind][page=2]}
      {\externalfigure[face-smile][location=middle] Hidden cause}
      {\externalfigure[fig-ind][page=3]}
      {\llap{\externalfigure[face-sad][location=middle]} Explanation }
    \stopcombination

  \head Blocking of a trail

    \vskip -.5\baselineskip
    A trail from $a$ to $b$ is blocked by $C$ if $\exists$ a node $v$ on the
    trail such that either:
    \vskip -.5\baselineskip
    \startitemize[1]
      \item either $\to v \to$, $\from v \from$, or $\from v \to$, and $v \in C$
      \item $\to v \from$ and neither $v$ nor any of $v$'s descendants are in
        $S$.
    \stopitemize
\stopitemize

\SlideTitle Conditional independence 

\startitemize

  \head d-separation

    $A$ is d-separated from $B$ by $S$ if all trails from $A$ to $B$ are blocked
    by $S$

  \head Conditional independence

    For any probability measure $P$ that factorizes according to a DAFG, 
    
    \blank[medium]
    \startalignment[middle]
      $A$ d-separated from $B$ by $C$ implies \crlf
      $X_A$ is conditionally independent
      of $X_B$ given $X_C$, $P$~a.s.
    \stopalignment

  \head Efficient algorithms to verify d-separation

    \vskip -1\baselineskip
    \startitemize
      \item Moral graph
        \hskip 5em
      \symbol[2]\space Bayes Ball
    \stopitemize

\stopitemize

\SlideTitle Automated Structural results

\startitemize
  \head First attempt

    \startitemize[unpacked]
      \item \emph{Dependent rewards:} $R_d(\tilde n) = R \cap
        \DESCENDANTS{\tilde n}$

      \item \emph{Irrelevant data:} At a control node $\tilde n$, and parent $i$
        is irrelevant if $R_d(\tilde n)$ is d-separate from $i$ given
        $\PARENTS{\tilde n} \cup \CHILDREN{\tilde n} \setminus \{i\}$

      \item \emph{Requisite data:} All parents that are not irrelevant
    \stopitemize

  \head Structural result

    \startitemize
      \item Without loss of optimality, we can choose $u_n =
        g_n($requisite$(\tilde n))$
    \stopitemize


\stopitemize
  
\SlideTitle 

\startshowresult{Structural Results for MDP --- Step 1}{1}
\startitemize
  \head Pick node $g_3$.

    \startitemize
      \item Original $u_3 = g_3(x_1, x_2, x_3, u_1, u_2)$
      \item requisite($g_3$) = $\{x_3\}$
      \item Thus, $u_3 = g_3(x_3)$
    \stopitemize
\stopitemize
\stopshowresult

\startshowresult{Structural Results for MDP --- Step 2}{2}
\startitemize
  \head Pick node $g_2$.

    \startitemize
      \item Original $u_2 = g_2(x_1, x_2, u_1)$
      \item requisite($g_2$) = $\{x_2\}$
      \item Thus, $u_2 = g_2(x_2)$
    \stopitemize
\stopitemize
\stopshowresult


\SlideTitle Structural Results for MDP --- Step 2

\placefigure[here,none]{}
  {\externalfigure[fig-mdp][page=3, width=0.8\textwidth]}

\Interlude Almost there \unknown{}

\SlideTitle A real-time source coding problem

\startitemize[paper]
  \item Hans S.\ Witsenhausen, 
    \emph{On the structure of real-time source coders}, \crlf
    Bell Systems Technical Journal, 
    vol 58, no 6, pp 1437-1451, July-August 1979
\stopitemize


\startitemize
  \head Mathematical Model

    \startitemize
      \item Source: First order Markov $f_t(x_{t+1}|x_t)$ 
      \item Real-time source coder: $y_t = c_t(x(1:t), y(1:t-1))$
      \item Finite memory decoder: $\hat x_t = g_t(y_t, m_{t-1})$
      \item \hphantom{Finite memory decoder:} $m_{t+1} = l_t(y_t, m_{t-1})$
      \item Cost: $ρ(d_t | x_t, \hat x_t)$
    \stopitemize

\stopitemize

\IncludePicture
  [horizontal]
  [fig-rt]
  {Graphical model for real-time communication}

\Interlude Need to take care of deterministic functions!

\SlideTitle Functionally determined nodes

\startitemize
  \head Functionally determined 

    \vskip -1.5\baselineskip
    \startitemize
      \item $X_B$ is functionally determined by $X_A$ if
        \emph{$X_B \INDEPENDENT X_N \mid X_A$}
    \stopitemize

    \head Conditional independence with functionally determined nodes

    \vskip -1\baselineskip
      \startitemize
        \item Can be checked using \important{D}-separation
        \item Similar to d-sep: in the defn of blocking change
          \quotation{in $C$} by \rlap{\quotation{is func detm by $C$}}
      \stopitemize

  \head Blocking of a trail (version that takes care of detm nodes)

    \vskip -.5\baselineskip
    A trail from $a$ to $b$ is blocked by $C$ if $\exists$ a node $v$ on the
    trail such that either:
    \vskip -.5\baselineskip
    \startitemize[1]
      \item either $\to v \to$, $\from v \from$, or $\from v \to$, and \emph{$v$ is functionally determined by $C$}
      \item $\to v \from$ and neither $v$ nor any of $v$'s descendants are in
        $S$.
    \stopitemize
\stopitemize

\SlideTitle Automated Structural results

\startitemize
  \head Second attempt

    \startitemize[unpacked]
      \item \emph{Irrelevant data:} Change d-separation by D-separation
        
      \item \emph{Requisite data:} All parents that are not irrelevant
    \stopitemize

  \head Structural result

    \startitemize
      \item Without loss of optimality, we can choose \crlf
        \centerline{$u_n =
        g_n($requisite$(\tilde n)$, \important{functionally\_detm($\tilde n$)
        $\cap$ ancestors($R_d(\tilde n)$)}$)$}
    \stopitemize
      \startimportant
        Proof: use policy independence of conditional expectation and
        follow the steps of the three step lemma.
      \stopimportant
\stopitemize

\IncludePicture
  [horizontal]
  [fig-rt]
  {Lets try this!}

% \IncludePicture
%   [horizontal]
%   [fig-rt-cycle]
%   [highlight=yes,
%   alternative=focus,
%   x=3.25,
%   y=4,
%   xscale=2.1,
%   yscale=2,
%   direction=-45]
%   {Can lead to cycles}
% 
% \SlideTitle Automated Structural results
% 
% \startitemize
%   \head Third attempt
% 
%     \startitemize[unpacked]
%       \item \emph{Effectively observed:}:
%         functionally\_determined($\tilde n) \setminus \ANCESTORS{R_d(\tilde
%         n)}$
%     \stopitemize
% 
%   \head Structural result
% 
%     \startitemize
%       \item Without loss of optimality, we can choose \crlf
%         \centerline{$u_n =
%         g_n($requisite$(\tilde n)$, \important{effective\_obs($\tilde n$)$)$}}
%     \stopitemize
% 
% \stopitemize
% 
% \SlideTitle Automated Structural results
% 
% \startitemize
%   \head Third attempt
% 
%     \startitemize[unpacked]
%       \item \emph{Effectively observed:}:
%         Functionally determined at $\tilde n \setminus \ANCESTORS{R_d(\tilde
%         n)}$
%     \stopitemize
% 
%   \head Structural result
% 
%     \startitemize
%       \item Without loss of optimality, we can choose \crlf
%         \centerline{$u_n =
%         g_n($requisite$(\tilde n)$, \important{effective\_obs($\tilde n$)$)$}}
%     \stopitemize
% 
%     \startimportant
%       Works! Proof: use policy independence of conditional expectation and
%       follow the steps of the three step lemma.
%     \stopimportant
% 
% \stopitemize

\dorecurse{14}{
\IncludePicture
  [horizontal]
  [fig-rt]
  [page=\the\numexpr#1+1]
  {Structural Results for Dec MDP --- Step #1}}

\SlideTitle Structural Results for real-time communication

\startitemize
  \head Graphically

    \vskip -2\baselineskip
    \placefigure[here,none]{}
      {\externalfigure[fig-rt][page=15, width=0.8\textwidth]}

    \head Mathematically

      \startitemize
        \item Original Encoder:
          $y_t = c_t(x_1, \dots, x_t, y_1, \dots, y_{t-1})$

        \item New encoder: 
          $y_t = c_t(x_t, m_{t-1})$

      \stopitemize
\stopitemize

\SlideTitle Automated Structural results

\startitemize
  \head Simplify Once

    \startitemize
      \item For each control node
        \startitemize
          \item Find irrelevant nodes and functionally determined nodes.
          \item Remove edges from irrelevant nodes, add edges from functionally
            determined nodes.
        \stopitemize
    \stopitemize

  \head Find fixed point

    \startitemize
      \item Keep on simplifying until the graph does not change
    \stopitemize

  \head Software Implementation

    \startitemize
      \item A EDSL to find structural results \hfill (is there times
        for a demo??)
    \stopitemize
\stopitemize

\Interlude Conclusion

\SlideTitle Conclusion

\startitemize
  \head First Model

    \startitemize
      \item Private and common observations
      \item The notion of a coordinator
      \item Sequential decomposition for finite and infinite horizon
    \stopitemize

  \head Second Model

    \startitemize
      \item Graphical model
      \item Automated structural results
    \stopitemize
\stopitemize

\SlideTitle Future directions

\startitemize
  \head Computational algorithms

    \startnarrower
      The theory will not be practical until efficient algorithms to solve
      nested optimality equations are identified
    \stopnarrower

  \head Graphical Model

    \startitemize
      \item Automated method to discover belief states
      \item Automated method for sequential decomposition
    \stopitemize

  \head Extensions to non-stochastic systems

    \startitemize
      \item Appear feasible
      \item Use $σ$-algebra rather than the measure induced by the $σ$-algebra
    \stopitemize

\stopitemize

\Interlude \important{Thank you}

\stoptext

\SlideTitle Reflections

\startitemize
  \head Going from one agent systems to two agent systems requires a paradigm
    shift 

  \item The success of POMDPs has led to the failure to understand multi-agent
    systems

    \startimportant
      Do not think in terms of data 
    \stopimportant

  \head Do not think in terms of \bold{one} agent, \crlf \null \hskip 8em
        think in terms of \bold{every} agent

\stopitemize

\stoptext
